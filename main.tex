\documentclass[a4paper,fleqn]{cas-dc}

\usepackage[numbers]{natbib}
\usepackage{amssymb}
\usepackage{amsmath}
\usepackage{mathtools}
\usepackage{booktabs}
\usepackage[utf8]{inputenc}
\usepackage{textcomp}

\usepackage{flushend}
\usepackage{algorithm}
\usepackage{algpseudocode}



\numberwithin{equation}{section}

\begin{document}
\let\WriteBookmarks\relax
\def\floatpagepagefraction{1}
\def\textpagefraction{.001}

\shorttitle{Intelligent Intrusion Detection Framework}
\shortauthors{S Gopikrishnan et~al.}

\title [mode = title]{Generative and Trust-Aware AI-SDN Routing for Intelligent and Underwater IoT Networks}                     

\author[1]{S Gopikrishnan}[orcid=0000-0001-9082-9012]
\ead{gopikrishnan.s@vitap.ac.in}
\address[1]{School of Computer Science and Engineering, VIT-AP University, Amaravathi. Andhra Pradesh, India}


\author[1]{Harish} [orcid=0000-0001-8097-2739]
\ead{harish.22bcb7135@vitapstudent.ac.in}







\cortext[cor1]{Corresponding author}

\begin{abstract}
The rise of Internet of Things (IoT) and Industrial IoT (IIoT) devices has increased the cyber attack surface, highlighting the need for scalable and adaptive intrusion detection systems. implementations.
The rise of Internet of Things (IoT) and Industrial IoT (IIoT) devices has increased the cyber attack surface, highlighting the need for scalable and adaptive intrusion detection systems. implementations.
The rise of Internet of Things (IoT) and Industrial IoT (IIoT) devices has increased the cyber attack surface, highlighting the need for scalable and adaptive intrusion detection systems. implementations.
The rise of Internet of Things (IoT) and Industrial IoT (IIoT) devices has increased the cyber attack surface, highlighting the need for scalable and adaptive intrusion detection systems. implementations.
The rise of Internet of Things (IoT) and Industrial IoT (IIoT) devices has increased the cyber attack surface, highlighting the need for scalable and adaptive intrusion detection systems. implementations.
\end{abstract}

\begin{keywords}
Intrusion Detection System (IDS) \sep
Internet of Things (IoT) \sep 
Edge Computing \sep 
Deep Learning \sep 
Feature Optimization \sep 
Synaptic Intelligence
\end{keywords}


\maketitle

\section{Introduction}
\label{sec:1}

The rapid growth in Internet of Things technology is helping many industries to bring smart type of automation inside control systems used in industry, farming areas, healthcare, and also transportation systems. 

This paradigm shift has enabled unprecedented levels of automation and data exchange in real-time applications such as healthcare, industrial automation, smart cities, and environmental monitoring \cite{Saadouni20248655}. Concurrently, the increasing complexity and heterogeneity of modern networks have spurred the adoption of Software-Defined Networking (SDN), which decouples the control and data planes to offer centralized control, enhanced programmability, and a network-wide perspective for efficient management and improved performance \cite{OspinaCifuentes2024}. The integration of IoT with SDN, known as SDN-IoT, provides a robust infrastructure for managing the vast and diverse data generated by IoT devices \cite{Zabeehullah2025}.

A critical enabler in this technological evolution is Artificial Intelligence (AI), particularly deep learning, which is adept at discerning complex patterns, facilitating anomaly detection, and enabling predictive capabilities \cite{Lin20231781}. Among AI advancements, Generative Artificial Intelligence (GenAI), including Generative Adversarial Networks (GANs), has emerged as a powerful tool for diverse applications, such as data generation and augmentation, traffic flow prediction, autonomous driving, anomaly detection, and routing optimization within Intelligent Transportation Systems (ITS) \cite{Rong2025}. Despite these significant opportunities, the expansive IoT ecosystem, coupled with SDN architectures, faces considerable challenges including data loss, security vulnerabilities, device failures, and sophisticated malicious attacks \cite{Yan2025}.

A particularly challenging environment for IoT deployment is Underwater Wireless Sensor Networks (UWSNs). The open and time-varying nature of underwater transmission environments makes data susceptible to interception, interference, and tampering by malicious nodes, which are difficult to identify and distinguish \cite{wang2024gtr}. Ensuring secure routing and reliable data forwarding in such dynamic and resource-constrained settings presents unique and complex problems that demand innovative solutions \cite{Fu2024}.


\subsection{Necessity}
The network security is becoming serious matter because of the fast growing number of IoT devices. Just using firewalls and encryption is not enough to give full protection to IoT systems from advanced cyber threats. 

The imperative for robust, adaptive, and trust-aware routing across intelligent IoT and UWSN systems is driven by several critical factors: 
\begin{itemize} 
\item Resource Constraints and Scalability: IoT devices are inherently resource-limited in terms of battery life, processing power, and communication bandwidth \cite{Abujassar2025}. In large-scale deployments, these constraints are exacerbated, making energy-efficient routing paramount for extending network lifespan \cite{Fu2024}. Traditional network protocols often struggle with the vast scale and heterogeneity of IoT, leading to inefficient resource utilization and scalability challenges \cite{Li2025}. 
\item Dynamic and Complex Environments: Modern IoT and ITS operate in highly dynamic environments where traffic patterns, network topology, and device behavior can change rapidly \cite{song2025emergency}. This dynamism necessitates routing protocols that can adapt in real-time to unexpected events, congestion, or link failures, ensuring low-latency communication and continuous service delivery, particularly for critical applications like emergency response in ITS \cite{Byakodi2023}. 
\item Security Vulnerabilities and Malicious Attacks: The distributed and open nature of IoT networks, including UWSNs, renders them highly susceptible to various cyber threats, such as data breaches, spoofing, denial-of-service (DoS) attacks, unauthorized access, and data manipulation \cite{Saadouni20248655}. In UWSNs, malicious nodes can easily blend into the network and exploit the challenging transmission conditions. Therefore, secure data transmission, anomaly detection, and defense against evolving threats are non-negotiable requirements . 
\item Data Integrity and Trustworthiness: Ensuring the integrity and confidentiality of data from source to destination is crucial, especially for sensitive information in healthcare or smart city applications \cite{Anantula20255162}. Trust management mechanisms are essential for identifying and isolating malicious or unreliable nodes, thereby safeguarding the network from compromised entities \cite{wang2024gtr}. 
\end{itemize} 
Generative AI offers a promising avenue to address these necessities by synthesizing realistic data for rare attack scenarios, modeling complex and unpredictable traffic patterns, and enhancing the detection of anomalies, thereby strengthening security and optimizing routing decisions in dynamic environments \cite{devi2024robust}.

\subsection{Problem Statement}
The rapidly increasing number of IoT devices in various industries is posing serious cybersecurity challenges, as standard security methods such as firewalls and encryption a


\subsection*{Problem Formulation}





\subsection{Objective}
The core idea behind the research work is to develop and construct one Intelligent Two-Layer Intrusion Detection System (IDS) for IoT which is maintaining good balance in between accuracy, efficiency and low cost. 

\subsection{Contributions}


A new two-layer intrusion detection structure named REGA-SACNN is proposed, which divides lightweight anomaly detection at the edge and deep, context-based classification in the fog layer. This setup enables real-time reaction while maintaining high classification performance.



\section{Background Study}


\begin{table*}[htbp]
\centering
\caption{Summary of Literature Reviews}
\label{tab:tab1}
\begin{tabular}{p{1cm} p{8cm} p{7cm}}
\toprule
\textbf{Paper} & \textbf{Advantages} & \textbf{Limitations} \\
\midrule
\cite{Kostas2025} & Highlights the risks of using individual packet features (IPF) in intrusion detection, demonstrating their poor generalization across datasets. & Models based on IPF show high detection rates in isolated cases but fail in cross-session testing, reducing real-world applicability. \\
\cite{chen2025sicnn} & Proposes a synaptic intelligence-based CNN that reduces catastrophic forgetting and improves performance in dynamic IoT environments. & Requires significant computational power for training and may not be suitable for resource-constrained IoT devices. \\
\bottomrule
\end{tabular}
\end{table*}


\section{Literature Review}

The work proposed by Abujassar et al.\cite{Abujassar2025} introduces the Interior Neighbors Route with Low Fault (INRwLF) algorithm, an AI-based, energy-efficient routing protocol for SDN-enabled IoT healthcare networks. It integrates PSO for adaptive clustering, ABC for load balancing, and ACO for shortest-path routing, managed by a centralized SDN controller. Trust-based neighbor evaluation improves security, while redundant primary/backup paths enhance fault tolerance. Advantages include up to 90\% longer network lifetime, reduced latency, low overhead, scalability, and resilience. Limitations include reduced stability in high-mobility conditions, testing limited to 300 nodes, absence of blockchain or advanced encryption, and dependence on a centralized SDN controller, creating a potential single point of failure.

The work proposed by Zouhri et al.\cite{Zouhri2025} introduces a CTGAN-ENN hybrid framework to enhance intrusion detection system (IDS) performance and interpretability. First, the model applies Edited Nearest Neighbor (ENN) undersampling to remove noisy majority-class samples, reducing overlap with minority classes. Then, Conditional Tabular GAN (CTGAN) generates realistic synthetic samples for minority classes to address imbalance. The balanced dataset undergoes feature selection using both filter- and wrapper-based techniques, followed by classification with RF, XGB, LGBM, and DNN models. Finally, explainable AI methods (SHAP, LIME, Global Surrogate) provide local and global interpretability. Compared with WGAN, WGAN-GP, SMOTE, and ADASYN, CTGAN-ENN achieved superior accuracy (up to 99.99\%) on CICIDS2018, CIC-ToNIoT, and NF-UNSW-NB15-v2 datasets. Pros: Effectively addresses class imbalance, reduces noise, improves generalization, and increases transparency via multi-method interpretability. Cons: Higher computational cost, possible GAN training instability, and reliance on dataset-specific tuning for optimal performance, which may affect scalability to entirely new intrusion scenarios.

The paper proposed by Souza et al.\cite{Souza2025} works on an SDN-based malware analysis and detection framework that uses the centralized control and global visibility of Software-Defined Networking to enhance network security. The model combines real-time traffic monitoring, deep packet inspection, and machine learning-based classification at the SDN controller to efficiently identify malicious activity. The pros include centralized control for rapid policy updates, flexible and dynamic rule deployment, improved detection accuracy through global traffic analysis, adaptability to evolving threats, and the ability to integrate diverse detection techniques. The cons include the risk of the controller becoming a performance bottleneck or single point of failure, dependence on high-quality training data for accuracy, and potential scalability issues in large-scale networks.

The work by Kah Phooi Seng et al.\cite{seng2022artificial} proposes an AI-enabled IoT framework for Underwater Wireless Sensor Networks (UWSNs), focusing on secure and energy-efficient communication. It integrates machine learning models for anomaly detection, trust management systems for malicious node identification, and optimization algorithms for energy-aware routing. By leveraging AI-driven decision-making, the model enhances packet delivery, prolongs network lifetime, and strengthens resilience against cyber threats in harsh underwater environments. Pros include improved security, adaptability to dynamic aquatic conditions, and optimized resource usage. Cons include high computational costs for AI models, dependency on reliable underwater communication infrastructure, and challenges in deploying resource-constrained nodes in real-world UWSN scenarios.

The paper by Salehiyan et al.\cite{Salehiyan2025} proposes an Optimized Transformer-GAN-AE hybrid framework for intrusion detection in Edge and IIoT systems. The model integrates a Transformer for sequence feature extraction, a Generative Adversarial Network (GAN) for synthesizing realistic intrusion data, and an Autoencoder (AE) for dimensionality reduction and anomaly detection. The training process includes hyperparameter optimization to improve performance across WUSTL-IIoT-2021, Edge-IIoTset, and TON-IoT datasets. The pros include enhanced detection accuracy through multi-component integration, robustness to imbalanced data, scalability to various IIoT environments, and adaptability to evolving attack patterns. The cons include high computational requirements, potential GAN training instability, and the need for extensive tuning to maintain consistent performance across heterogeneous datasets.

The paper by Anantula et al.\cite{Anantula20255162} proposes a GAN-based network intrusion detection approach aimed at improving detection accuracy and handling class imbalance in cybersecurity datasets. The model leverages a Generative Adversarial Network to generate synthetic attack samples, thereby enriching the minority classes, followed by classification using machine learning algorithms. The pros include improved detection rates for rare attack types, better generalization through data augmentation, adaptability to evolving threats, and the ability to enhance classifier training without manually collecting more data. The cons include high computational cost, potential instability in GAN training, risk of generating unrealistic samples that may mislead the classifier, and dependence on the representativeness of the original dataset.

The paper proposed by Agitha W et al.\cite{W2025} proposes a GAN-based IoT malware detection framework that addresses data scarcity and imbalance in IoT security datasets. The model employs a Generative Adversarial Network to synthesize realistic IoT malware traffic samples, which are then combined with real data to train deep learning classifiers for accurate detection. The pros include improved detection accuracy for underrepresented malware classes, better classifier generalization, adaptability to new IoT threats, and reduced reliance on large-scale real-world data collection. The cons include high computational requirements, potential instability during GAN training, the possibility of generating low-quality or non-representative samples, and dependence on dataset-specific characteristics for optimal performance.

The paper proposed by Enerst Edozie et al.\cite{Edozie2025} introduces an AI-driven anomaly detection framework for telecom networks that integrates machine learning and deep learning techniques to detect unusual patterns in network traffic. The model utilizes supervised, unsupervised, and hybrid approaches, incorporating methods such as deep neural networks, autoencoders, and clustering algorithms to improve detection accuracy across diverse anomaly types. The pros include enhanced adaptability to evolving telecom threats, scalability to large network infrastructures, the ability to handle complex and high-dimensional data, and improved accuracy through multi-technique integration. The cons include high computational resource requirements, the need for large and representative training datasets, potential false positives in dynamic network conditions, and challenges in real-time deployment for high-throughput telecom environments.

The paper proposed by Prasanth et al.\cite{Prasanth2025} proposes an SDN–cloud-based framework for remote neurostimulator health monitoring, aiming to enhance patient care through improved connectivity, scalability, and security. The model uses Software-Defined Networking to provide centralized control and dynamic traffic management, while cloud integration enables real-time data storage, processing, and analytics. This architecture supports seamless communication between neurostimulators, healthcare providers, and cloud servers, ensuring timely alerts and medical interventions. The pros include flexible network management, scalability for large patient bases, reduced latency for critical health data, and enhanced security through programmable policies. The cons include dependency on reliable internet connectivity, potential risks of a single point of failure in the SDN controller, high initial setup costs, and challenges in meeting strict medical data privacy regulations across regions.

The paper proposed by Victor G. da Silva Ruffo et al.\cite{daSilvaRuffo2025} introduces a GAN-based intrusion and anomaly detection model for IP flow-based networks that leverages the generative capabilities of GANs to learn normal traffic patterns and identify deviations as potential threats. The model uses the generator to create synthetic benign traffic and the discriminator to distinguish between real and generated data, enabling the detection of anomalies without requiring labeled attack data. The pros include the ability to operate in unsupervised settings, improved detection of unknown or zero-day attacks, adaptability to evolving network behaviors, and reduced dependence on large labeled datasets. The cons include high computational demands, possible instability in GAN training, sensitivity to hyperparameter tuning, and the risk of reduced accuracy in highly dynamic or noisy traffic environments.

The paper proposed by sefati et al.\cite{Sefati2025} introduces us to an intelligent congestion control framework for Wireless Sensor Networks (WSNs) using GANs and optimization algorithms. The model employs a Generative Adversarial Network to generate realistic traffic patterns, which help in predicting and managing congestion more effectively. Optimization algorithms are then applied to dynamically allocate resources and balance network load. The pros include improved congestion prediction accuracy, efficient resource utilization, adaptability to varying traffic conditions, and enhanced overall network performance. The cons include high computational overhead for sensor nodes with limited resources, potential instability during GAN training, increased energy consumption, and dependency on carefully tuned optimization parameters for consistent results.

The paper proposed by Li et al.\cite{Li2025} gives us an introduction to a SDN-based security framework for IoT networks that leverages the programmability and centralized management of SDN to detect and mitigate cyberattacks. The model integrates machine learning techniques at the SDN controller to analyze IoT traffic patterns in real time, enabling dynamic policy enforcement and rapid threat response. The pros include centralized traffic visibility, flexible rule updates, improved detection accuracy through learning-based analysis, scalability to diverse IoT environments, and adaptability to emerging threats. The cons include the risk of the controller being a single point of failure, high computational overhead for continuous traffic analysis, dependency on training data quality, and potential latency issues in large-scale IoT deployments. 

The paper proposed by Jianhui Lv et al.\cite{Lv202523536} introduces an AI-driven resource management framework for energy-efficient aerial computing in large-scale healthcare SDN–IoT systems. The model integrates Software-Defined Networking for centralized control, IoT for patient monitoring, and aerial computing (such as UAVs) for data collection and processing. Artificial intelligence techniques are applied for resource allocation, energy optimization, and workload balancing across the heterogeneous infrastructure. The pros include improved energy efficiency, scalability for large healthcare networks, enhanced flexibility through AI-based decision-making, and reduced latency in critical healthcare applications. The cons include dependency on reliable UAV operations, high computational requirements for AI models, risks of SDN controller bottlenecks, and challenges in maintaining secure and privacy-compliant healthcare data transmission.

The paper proposed by Zabeehullah et al.\cite{Zabeehullah2025} proposes a secure AI-based framework for intelligent traffic prediction and routing in SDN-enabled Consumer IoT environments. The model integrates artificial intelligence with Software-Defined Networking to predict network traffic patterns and optimize routing decisions dynamically, while also embedding security mechanisms to defend against cyberattacks. By leveraging centralized SDN control and AI-driven learning, the system ensures efficient bandwidth usage, low latency, and secure data transmission across IoT devices. The pros include improved traffic management, enhanced security, scalability for large IoT networks, and adaptability to dynamic traffic conditions. The cons include reliance on high-quality training data, potential controller bottlenecks as a single point of failure, high computational overhead for AI models, and challenges in ensuring real-time performance in large-scale deployments.

The paper proposed by Zhi-Xian Zheng et al.\cite{Zheng20251011} proposes an IoT intrusion detection model that combines Generative Adversarial Networks (GANs) with Transformer neural networks to improve detection performance and address data imbalance issues. GANs are employed to generate synthetic samples for underrepresented attack types, while the Transformer is used for sequence modeling and feature extraction, enabling more accurate intrusion classification. The pros include improved detection accuracy for minority classes, robustness to imbalanced datasets, strong feature representation through the Transformer, and adaptability to evolving IoT attack patterns. The cons include high computational cost, potential instability in GAN training, the need for large-scale tuning of Transformer parameters, and possible scalability issues in real-time IoT deployments.

The paper proposed by Fawad Naseer et al.\cite{Naseer2025} proposes an adaptive AI-powered elderly care framework that integrates Generative Adversarial Networks (GANs) with IoT in smart homes. The model employs IoT devices to continuously monitor elderly health and behavior data, while GANs generate synthetic data to address scarcity and improve model training for personalized care. This adaptive system enables anomaly detection, health risk prediction, and personalized assistance tailored to individual needs. The pros include enhanced personalization in elderly care, improved anomaly detection accuracy, better handling of limited training data, and adaptability to evolving health conditions. The cons include high computational requirements, potential privacy concerns in sensitive health data, instability during GAN training, and dependency on reliable IoT infrastructure for continuous monitoring and real-time response.


The paper proposed by Ahmad Almadhor et al.\cite{Almadhor2025} gives an introduction to Generative AI-driven context-aware BDI-based smart routing protocol for Intelligent Transportation Systems (ITS). The model combines Generative AI with the Belief-Desire-Intention (BDI) agent framework to enable adaptive and intelligent routing decisions based on real-time traffic, environmental context, and predictive analysis. By generating synthetic traffic scenarios, the system improves decision-making under uncertain or sparse data conditions. The pros include improved traffic flow management, adaptability to dynamic road conditions, enhanced decision-making through BDI reasoning, and robustness in handling incomplete or imbalanced data. The cons include high computational demands, reliance on accurate contextual data, potential scalability issues in large urban ITS deployments, and the complexity of integrating BDI agents with generative AI in real-time environments.

The paper proposed by Tingxuan Fu et al.\cite{Fu2024} introduces an energy-aware secure routing scheme for IoT networks using two-way trust evaluation. The model integrates trust-based mechanisms with energy-efficient routing to ensure both secure and reliable data transmission while prolonging the network lifetime. Trust values are calculated based on node behavior and interactions, while routing decisions are optimized to balance energy consumption across nodes. The pros include enhanced network security through trust evaluation, improved reliability in data delivery, extended lifetime of IoT networks via energy-aware design, and resilience against malicious or selfish nodes. The cons include additional computational and communication overhead for continuous trust evaluation, potential delays in real-time scenarios, dependence on accurate trust metrics, and scalability challenges in very large IoT environments.

The paper proposed by Swathi et al.\cite{Swathi202410653} inroduces a GAN-based traffic anomaly detection framework for SDN environments. The model leverages the generative capability of GANs to learn normal traffic patterns and detect deviations as anomalies, while the centralized SDN controller provides global visibility and facilitates rapid deployment of detection and mitigation rules. This integration allows efficient handling of dynamic and large-scale traffic flows in modern networks. The pros include improved accuracy in detecting unknown or zero-day anomalies, adaptability to evolving traffic behaviors, centralized control for fast response, and reduced dependence on labeled attack data. The cons include high computational overhead, instability risks in GAN training, reliance on the SDN controller as a single point of failure, and challenges in ensuring real-time detection for high-throughput networks.

The paper proposed by Alnaser et al.\cite{Alnaser2024} proposes an AI-driven framework for optimizing multi-tier scheduling and secure routing in edge-assisted Software-Defined Wireless Sensor Networks (SDWSNs) using Moving Target Defense (MTD). The model leverages SDN for centralized control, AI algorithms for adaptive scheduling and routing, and MTD techniques to dynamically alter network configurations, thereby reducing the attack surface. This integration enhances both performance and security in resource-constrained WSN environments. The pros include improved resource utilization, enhanced resilience against cyberattacks, flexibility in adapting to dynamic conditions, and scalability for large deployments. The cons include high computational overhead for AI and MTD operations, reliance on continuous controller availability, potential latency in real-time applications, and complexity in implementing MTD strategies in practical large-scale SDWSNs.

The paper proposed by Zacaron et al.\cite{Zacaron2024} introduces Generative Adversarial Network (GAN) models for anomaly detection in Software-Defined Networks (SDNs). The model leverages GANs to learn the distribution of normal network traffic and detect anomalies as deviations from this learned pattern. By utilizing the centralized control and global visibility of SDN, the GAN-based IDS can efficiently monitor traffic and respond to malicious activities. The pros include the ability to detect unknown or zero-day attacks, reduced reliance on labeled datasets, adaptability to evolving threats, and centralized control for quick mitigation. The cons include high computational complexity, instability risks during GAN training, reliance on the SDN controller as a single point of failure, and potential delays in detection under high-throughput network conditions.

The paper proposed by Wang et al.\cite{Wang2024427} proposes a data-driven cooperative bus route planning system using Generative Adversarial Networks (GANs) and metric learning. The model employs GANs to generate realistic passenger flow and traffic pattern data, while metric learning enhances route optimization by measuring similarities in travel demand across different routes. This combination enables more efficient and adaptive bus scheduling that responds to real-time urban mobility needs. The pros include improved route optimization accuracy, adaptability to dynamic traffic conditions, reduced passenger waiting times, and enhanced resource utilization in public transport systems. The cons include high computational requirements, dependence on accurate and diverse input data, potential instability during GAN training, and scalability challenges when applied to very large metropolitan transportation networks.

The paper proposed by Li et al.\cite{Li2023} proposes a deep reinforcement learning (DRL)-based path selection scheme for detecting malicious behavior in SDN/NFV-enabled networks. The model leverages DRL to dynamically select secure and efficient network paths by learning from traffic patterns and feedback, while SDN provides centralized control and NFV enables flexible deployment of virtualized security functions. The pros include improved detection accuracy of malicious behaviors, adaptive and intelligent path selection, scalability in complex network environments, and efficient use of network resources. The cons include high computational overhead for training DRL models, reliance on continuous feedback for effective learning, potential delays in real-time decision-making, and risks associated with SDN controller dependency as a single point of failure.

The paper proposed by Phu et al.\cite{Phu2023} introduces a deep learning-based defense framework against packet injection attacks in Software-Defined Networks (SDNs). The model leverages the centralized control of SDN to collect real-time traffic features, which are then analyzed by deep learning classifiers to distinguish between legitimate and malicious packets. By learning complex traffic patterns, the system can detect stealthy injection attacks that traditional rule-based methods may miss. The pros include improved detection accuracy, adaptability to evolving attack strategies, real-time monitoring through SDN visibility, and reduced reliance on manual signature updates. The cons include high computational costs for training and inference, potential latency in high-throughput environments, dependence on high-quality training data, and vulnerability of the SDN controller as a single point of failure.

The paper proposed by Byakodi et al.\cite{Byakodi2023} introduces a Deep Reinforcement Learning-based routing optimization model (DeepQoSR) using Advantage Actor-Critic (A2C) networks to enhance packet routing in Software-Defined Networks (SDNs). The model integrates SDN with reinforcement learning, where the RL agent dynamically learns optimal routes based on Quality of Service (QoS) parameters such as bandwidth, packet loss, and latency. The approach outperforms traditional algorithms like Dijkstra’s shortest path in throughput and average delay by enabling self-learning and adaptive routing in data center networks. The pros include improved throughput, reduced latency, dynamic adaptability to network conditions, and scalability for large topologies. The cons include high computational overhead for training RL agents, dependency on large-scale simulations for accuracy, potential instability during training, and challenges in real-time deployment due to controller latency and resource constraints. 

The paper proposed by Aruna et al.\cite{devi2024robust} gives introduction to a Bio-Inspired Protocol using GANs (BIP-GANs) for secure and efficient IoT data transmission. The model integrates Generative Adversarial Networks (GANs) for anomaly detection, an Artificial Immune System (AIS) for identifying malicious activities, and Hybrid Automatic Repeat Request (HARQ) for error recovery, while queue learning ensures optimal routing. The approach improves robustness, reduces energy consumption, and extends network lifespan compared to existing protocols like EAP-IFBA and DNN-CSO. Pros include strong anomaly detection, adaptive routing, and improved reliability under dynamic IoT conditions. Cons include high computational requirements of GANs, increased complexity, and uncertain scalability in real-world IoT deployments

The paper proposed Zabeehullah et al.\cite{khan2024secure} introduces a Secure and Efficient AI-SDN-based Routing model for Healthcare-Consumer IoT (H-CIoT), combining Generative Adversarial Networks (GANs) for anomaly detection with Deep Reinforcement Learning (DRL) for adaptive routing. The model leverages SDN controllers to dynamically optimize routing policies while detecting imbalance data security attacks. Compared to OSPF, it improves throughput by 12\%, reduces latency by 20\%, and enhances resilience against minor-class malicious attacks. Pros include adaptive routing, improved QoS, enhanced security, and suitability for dynamic healthcare environments. Cons include computational complexity of GANs and DRL, reliance on extensive training data, and potential challenges in real-time scalability.

The paper proposed by Shijun Song et al.\cite{song2025emergency} proposes an Emergency Routing Protocol for Intelligent Transportation Systems (ITS) that integrates a Belief-Desire-Intention (BDI) framework with Generative AI and IoT sensors to optimize real-time emergency vehicle routing. Unlike static rule-based systems, the protocol adapts dynamically to traffic, congestion, and environmental conditions, achieving 95\% route accuracy, 95\% collision avoidance, and reduced latency (105 ms). Pros include real-time adaptability, improved emergency response efficiency, energy optimization, and resilience through IoT integration. Cons include the computational overhead of generative AI, dependency on sensor infrastructure, limited validation outside simulations, and potential challenges in scaling across diverse urban transportation networks. 

The paper proposed by Ni Zhang et al.\cite{zhang2025adversarial} introduces an Adversarial Generative Flow Network (AGFN) framework to solve Vehicle Routing Problems (VRPs), including CVRP and TSP. Unlike Transformer-based solvers, AGFN integrates Generative Flow Networks (GFlowNets) for generating diverse routing solutions and a discriminator to evaluate them in an adversarial manner, enhanced by a hybrid decoding method. Experiments show AGFN outperforms neural and heuristic baselines in solution quality and generalization to larger instances. Pros include scalability, diversity of solutions, adaptability, and improved generalization. Cons include computational overhead from adversarial training, reliance on extensive simulations, and possible inefficiency in resource-constrained or real-time routing scenarios.

The work by Bin Wang et al. \cite{wang2024gtr} proposes GTR (GAN-Based Trusted Routing) for Underwater Wireless Sensor Networks (UWSNs), integrating Generative Adversarial Networks (GANs) for trust evaluation with Q-Learning for adaptive routing. By constructing trust feature profiles, GTR detects malicious nodes even under unlabeled or imbalanced data and selects optimal forwarding routes. Simulations show higher packet delivery (+11\%), lower energy tax (-11.4\%), and increased throughput (+20.4\%) compared to baselines. Pros include strong malicious node detection, adaptability to dynamic underwater environments, and improved efficiency. Cons include computational demands of GANs, dependency on sufficient trust features, and challenges in deployment on energy-constrained underwater nodes.

\section{Proposed REGA-SACNN Methodology}


\section{Result Analysis}

\subsection{Experimental Setup}

In order to prove the utility of the suggested REGA-SACNN framework, an extensive list of tests was performed with publicly visible intrusion detection data sets. 


\subsection{Dataset}

https://ieee-dataport.org/datasets


\subsection{Result Analysis on X-IIoTID Dataset}




















\section{Introduction}

\subsection{Problem Statement}
Across the surveyed works, a common unresolved problem emerges:  
\emph{How to design a unified, secure, and adaptive routing framework that supports heterogeneous IoT environments (healthcare, transportation, and underwater networks), while simultaneously ensuring low-latency, resilience against malicious nodes, scalability, and trustworthy decision-making.}  

Specifically:  
\begin{itemize}
    \item AI-SDN routing in healthcare IoT enhances QoS but struggles with adversarial attacks, privacy, and real-time adaptability.  
    \item Emergency ITS routing with BDI and generative AI improves emergency response but lacks scalability and interoperability for large-scale deployments.  
    \item Neural VRP solvers (e.g., adversarial GFlowNets) optimize vehicle routing but remain computationally heavy and hard to generalize in safety-critical IoT settings.  
    \item GAN-based trust routing improves malicious node detection in underwater WSNs, but energy, imbalance in training data, and cross-domain transferability remain unresolved.  
\end{itemize}  

Thus, the open gap is the absence of an \textbf{integrated generative and trust-aware AI-SDN routing framework} that unifies:  
(i) predictive generative intelligence for dynamic route selection,  
(ii) adversarial robustness against malicious routing behaviors,  
(iii) SDN-enabled scalability and interoperability, and  
(iv) trust evaluation embedded into adaptive routing for mission-critical IoT domains.  

\subsection{Contributions}
The main contributions of this proposed work, \textbf{SecureRouteX}, are as follows:
\begin{enumerate}
    \item \textbf{Generative-Adversarial Routing Engine:} We develop a hybrid routing core that fuses BDI reasoning with Adversarial GFlowNets and GANs, enabling proactive and resilient route generation across healthcare, ITS, and underwater IoT networks.  
    \item \textbf{Trust-Embedded SDN Control:} A lightweight GAN-based trust evaluation is integrated into SDN controllers, ensuring that only reliable nodes participate in packet forwarding under dynamic conditions.  
    \item \textbf{Cross-Domain Adaptive Optimization:} We introduce a multi-objective learning layer balancing latency, energy, and trust scores, allowing seamless adaptation across urban ITS, healthcare IoT, and underwater WSN deployments.  
    \item \textbf{Resilient Federated Training:} We design a federated reinforcement learning pipeline where edge domains (e.g., hospital gateways, RSUs, buoy sinks) collaboratively refine routing policies without raw data sharing, preserving privacy.  
    \item \textbf{Scalable Evaluation Framework:} We provide cross-domain benchmarks with simulations (NS-3, Mininet-SDN, Aqua-Sim) and real datasets, evaluating routing efficiency, malicious detection accuracy, energy savings, and emergency response times.  
\end{enumerate}

\section{Related Work}
\subsection{AI-SDN Routing in Healthcare IoT}
\subsection{Emergency Routing in ITS}
\subsection{Neural and Generative Solvers for VRPs}
\subsection{Trust Management in UWSN Routing}

\section{System Model and Preliminaries}
\subsection{IoT Domains Considered}
\subsection{Network and Trust Assumptions}
\subsection{Adversarial and Dynamic Threats}

\section{Problem Formulation}
\subsection{Objectives}
\subsection{Constraints}
\subsection{Optimization Challenges}

\section{Proposed Method: SecureRouteX}
\subsection{Generative-Adversarial Routing Engine}
\subsection{Trust-Embedded SDN Control Layer}
\subsection{Cross-Domain Adaptive Optimization}
\subsection{Federated Training and Deployment}

\section{Theoretical Analysis}
\subsection{Convergence and Robustness Guarantees}
\subsection{Complexity Considerations}

\section{Evaluation}
\subsection{Experimental Setup}
\subsection{Baselines and Metrics}
\subsection{Results}
\subsection{Ablation Studies}

\section{Discussion}

\section{Conclusion and Future Work}








\section{Conclusion}

Overall, this article is presenting REGA-SACNN as a novice kind of a two-layer intrusion detection framework exclusively created to be used on IoT protection. 


\section*{Declarations}
\subsection*{Ethical Approval}
This manuscript reports studies that do not involve human participants, human data, human tissue, or animals.

\subsection*{Conflict of Interest}
The authors have no conflict of interest to declare that are relevant to the content of this article.

\subsection*{Authors' Contributions}
S. Gopikrishnan contributed to the conceptualization, methodology, software implementation, and dataset curation. Noufal was responsible for formal analysis, drafting the original manuscript, and designing the experiment protocols. Marco Rivera contributed to conceptualization, supervision, and in-depth analysis of the experimental results. Patrick Wheeler handled the review and editing of the final draft, as well as the overall validation of the study results.


\subsection*{Funding}
The authors did not receive financial support from any organization for the submitted work.

\subsection*{Availability of data and materials}
Data sharing is not applicable to this article, as no new data was created or analyzed in this study.

\subsection*{Acknowledgment}
We acknowledge that the hardware utilized in this research for evaluation is a part of Intel IoT Center for Excellence, VIT-AP University.  The authors also appreciate the support provided by the National Research and Development Agency (ANID) through the FONDECYT Regular grant number 1220556 and SERC Chile FONDAP 1523A0006. Additional funding was provided by the Research Project PINV01-743 of the National Council of Science and Technology (CONACYT). Furthermore, the authors acknowledge the International Research Collaboration Fund 2024-2025 from the University of Nottingham A7C200 and Programa de Redução de Assimetrias na Pós-Graduação (PRAPG) – Edital nº 14/2023 - DRI - CAPES. ID Number: 046.821.818-15. 

\bibliographystyle{cas-model2-names}
\bibliography{references}


\end{document}

